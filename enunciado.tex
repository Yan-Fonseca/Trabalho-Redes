\documentclass[12pt]{article}

\usepackage{xcolor}
\definecolor{darkgreen}{RGB}{0, 150, 0}

\begin{document}

\section*{Trabalho de Redes - 2024.3 - UFJF}

Construir um protocolo confiável de transporte sobre o protocolo não confiável UDP. Qualquer linguagem de programação pode ser usada.

\vspace{2em}

Para este trabalho você terá uma comunicação ponto a ponto (cliente e servidor) utilizando um UDP modificado para troca de pacotes. A implementação deve ser realizada utilizando o protocolo de transporte UDP, mas com a lógica de controle do seu protocolo estará no nível da camada de aplicação. Não é necessário modificar o protocolo UDP em si, mas sim colocar tratamentos dos dados enviados e recebidos.

São requisitos deste protocolo ter as seguintes funcionalidades:

\begin{enumerate}
    \item \textcolor{darkgreen}{Entrega ordenada para aplicação baseado na ordem dos pacotes (ter \# de sequência)}.
    \item \textcolor{darkgreen}{Confirmação acumulativa (ACK acumulativo) do destinatário para o remetente.}
    \item \textcolor{blue}{Adicione no protocolo um controle de fluxo, onde o remetente deve saber qual o tamanho da janela do destinatário, a fim de não afogá-lo.}
    \item \textcolor{blue}{Agora crie uma equação de controle de congestionamento, a fim de que, se a rede estiver apresentando perda (muitos pacotes com ACK pendentes, ACK duplicados ou timeout), ele deve ser utilizado para reduzir o fluxo de envio de pacotes.}
    \item Comentário 4.1: Você deve propor um controle de congestionamento, que pode ser baseado em algum existente no TCP, no QUIC ou qualquer outro protocolo. 
    \item Comentário 4.2: Lembre da Aula 13, onde há controle de congestionamento no TCP que utiliza uma janela "cwnd" e um variável "ssthresh" para controle das fases de "Slow Start" e "Congestion Avoidance".
    \item Avalie seu protocolo sobre 1 remetente (cliente) que envia um arquivo (ou dados sintéticos que preencham o payload) para 1 destinatário (servidor). 
    \item Restrição 5.1: Esses dados devem ser equivalente a, pelo menos, 10.000 pacotes. 
    \item Restrição 5.2: Para avaliar o controle de congestionamento, insira perdas arbitrárias (ou utilizando uma função rand()) de pacotes no destinatário (você pode fazer isso sorteando a cada chegada de um novo pacote se ele será contabilizado e processado ou descartado).
\end{enumerate}

\end{document}